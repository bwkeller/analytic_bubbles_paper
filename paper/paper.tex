\documentclass[linenumbers,twocolappendix,twocolumn]{aastex631}
\usepackage[T1]{fontenc}
\usepackage{ae,aecompl}
\usepackage{macros}
\usepackage{graphicx}
\usepackage{amssymb}
\usepackage{amsmath}
\usepackage{placeins}
\usepackage{times}

\def\spose#1{\hbox to 0pt{#1\hss}}
\def\lta{\mathrel{\spose{\lower 3pt\hbox{$\mathchar"218$}}
     \raise 2.0pt\hbox{$\mathchar"13C$}}}
\def\gta{\mathrel{\spose{\lower 3pt\hbox{$\mathchar"218$}}
     \raise 2.0pt\hbox{$\mathchar"13E$}}}
\def\inv{{${}^{-1}$}}

\newcommand{\sech}{\mathrm{sech} \,}

\newcommand{\appropto}{\mathrel{\vcenter{
  \offinterlineskip\halign{\hfil$##$\cr
    \propto\cr\noalign{\kern2pt}\sim\cr\noalign{\kern-2pt}}}}}


\shorttitle{Superbubbles Paper}
\shortauthors{J.S. Moreno \& B.W. Keller}

\begin{document}
\title{A Cool Paper About Superbubbles}

\author{J.S. Moreno}
\affiliation{Department of Physics and Materials Science, University of Memphis,
\\ 3720 Alumni Avenue, Memphis, TN 38152, USA}
\email{jsvdrmrn@memphis.edu}
\author[0000-0002-9642-7193]{B.W. Keller}
\affiliation{Department of Physics and Materials Science, University of Memphis,
\\ 3720 Alumni Avenue, Memphis, TN 38152, USA}

\begin{abstract} 
    My cool abstract
\end{abstract}

\keywords{keywords}

\section{Introduction}
\section{Conclusion}

\bibliographystyle{aasjournal}
\bibliography{references}

\appendix
\section{Important Timescales}
\subsection{Radiative Cooling of the Hot Interior}
When the hot interior of a bubble cools will be critical in determining when a
bubble transitions from an adiabatic, energy conserving evolution to a momentum
conserving one.  A number of different works have determined different values
for this, and we provide a brief survey of them as follows.

\citet{MacLow1988} provides in their equation 14 an approximation based on the
analytic calculations of \citet{Weaver1977}:

\begin{equation}
    t_{cool} =
    (16\Myr)\left(\frac{Z}{Z_\odot}\right)^{-35/22}L_{38}^{3/11}n_0^{-8/11}
\end{equation}

Where the ambient density $n_0$ is given in $m_p \hcc$, and $L_{38}$ is the
mechanical luminosity in $10^{38}\ergs$.  This will of course need to be
modified when $t_{cool}<t_{SN}$, because after the last SN has detonated the
energy injection halts.

\subsection{Timescales for Bubble Destruction}
Naturally, a bubble blown in the ISM will not persist forever, or else the
entire ISM of a galaxy will be evacuated after only a few star clusters have
formed. 

In \citep{Orr2022}, it is assumed a bubble fragments (or at least stops growing)
once the velocity of the shell is equal to the turbulent ISM velocity dispersion
$v=\sigma$.

\end{document}   
